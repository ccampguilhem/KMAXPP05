\documentclass{article}
\pagestyle{empty}
\usepackage{amsmath}
\usepackage{amsfonts}
\usepackage{amssymb}
\usepackage{amsthm}
\usepackage[french]{babel}
\usepackage{stmaryrd}
%\usepackage[latin1]{inputenc}
\setlength{\oddsidemargin}{-0.54cm}
\topmargin -2.54cm
\textwidth 17cm
\textheight 27cm
\newcommand{\BBr}{\mathbb{R}}
\newcommand{\BBn}{\mathbb{N}}
\newcommand{\BBp}{\mathbb{P}}
\newcommand{\BBe}{\mathbb{E}}
\newcommand{\Xn}{X_{n}}
\newcommand{\XXn}{X_{1},\ldots,\Xn}
\newcommand{\usn}{\frac{1}{n}}
\newcommand{\ust}{\frac{1}{3}}
\newcommand{\prs}[2] {\langle{#1},{#2}\rangle}
\newcommand{\Xtr}{X_{\text{train}}}
\newcommand{\Xts}{X_{\text{test}}}
\newcommand{\ytr}{y_{\text{train}}}
\newcommand{\yts}{y_{\text{test}}}
\newcommand{\lb}{\llbracket}
\newcommand{\rb}{\rrbracket}
%
\DeclareMathOperator{\logit}{logit}
%
\newcommand{\fonction}[5]{
 #1: \begin{array}{rcl}
	  #2 & \longrightarrow & #3 \\
     #4 & \longmapsto & #5 \end{array}
    }
%
\begin{document}
\noindent {\sc L FLEX}
\hfill 2023--2024\\
\noindent Introduction au Machine Learning 
\hfill Prénom\hspace{2cm} Nom\\
\vspace{1cm}

\begin{center}
{\bf\large \'Epreuve du 11 avril 2024} \\
{\it Durée 30 minutes. Les notes de cours et les calculatrices sont autorisées. Composer directement sur le sujet.}
\end{center}
%
\section*{Classifieur Logistique}
\noindent
On rappelle que la fonction logistique $\sigma$ est définie pour $t\in\BBr$ par
$\sigma(t)=\frac{1}{1+e^{-t}}$. 
On considère le classifieur binaire par régression logistique $\hat{y} : \BBr \rightarrow \{0,1\}$, de paramètres $\beta = 0.5$ et $\alpha = -0.5$, défini par
$$\hat{y} (x) = \begin{cases} 1 & \mbox{si } \sigma(\alpha +  \beta x ) \geq 1/2 \\ 0 & \mbox{sinon,} \end{cases}$$
On considère l'échantillon $\Xts, \yts$ suivant :

\[\Xts = 
\begin{pmatrix}
x^{(1)} \\
x^{(2)} \\
x^{(3)} \\
x^{(4)} 
\end{pmatrix}
=
\begin{pmatrix}
-2\\
0\\
0.5\\
2
\end{pmatrix} \qquad \yts = \begin{pmatrix}
0 \\
0 \\
1 \\
1 
\end{pmatrix}.
\]

\begin{enumerate}
\item Montrer que l'on peut calculer $\hat{y}$ par la règle suivante~: 
$$\hat{y} (x) = \begin{cases} 1 & \mbox{si } \alpha +  \beta x  \geq 0 \\ 0 & \mbox{sinon.} \end{cases}$$
\item Prédire les classes $\hat{y} (x^{(i)})$ pour $1 \leq i \leq 4$, puis calculer le taux d'erreur sur $\Xts$.
\item Le seuil de probabilités par défaut du classifieur par régression logistique est $1/2$, mais on peut le faire varier en définissant pour un seuil $\lambda \in [0,1]$ le classifieur 
\[\hat{y}_{\lambda} (x) = \begin{cases} 1 & \text{si } \sigma(\alpha + \beta x  ) \geq \lambda \\ 0 & \text{sinon.} \end{cases}\]
%
Trouver un seuil $\lambda \in [0,1]$ tel que le taux d'erreur associé à $\hat{y}_\lambda$ sur $\Xts$ soit nul.
\end{enumerate}


%
\end{document}
