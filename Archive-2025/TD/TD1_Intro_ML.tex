\documentclass{article}
\pagestyle{empty}
\usepackage{amssymb}
\usepackage[french]{babel}
%\usepackage[latin1]{inputenc}
\setlength{\oddsidemargin}{-0.54cm}
\topmargin -2.54cm
\textwidth 17cm
\textheight 27cm
\newcommand{\BBr}{\mathbb{R}}
\newcommand{\BBn}{\mathbb{N}}
\newcommand{\BBp}{\mathbb{P}}
\newcommand{\BBe}{\mathbb{E}}
\newcommand{\Xn}{X_{n}}
\newcommand{\XXn}{X_{1},\ldots,\Xn}
\newcommand{\usn}{\frac{1}{n}}
\newcommand{\ust}{\frac{1}{3}}
\newcommand{\prs}[2] {\langle{#1},{#2}\rangle}
%
\begin{document}
\noindent {\sc L FLEX}
\hfill 2024--2025\\
\noindent Introduction au Machine Learning 
\vspace{1cm}

\begin{center}
{\bf\large Feuille d'exercice 1} \\
{\it Modèle linéaire gaussien}
\end{center}
%


\section{Un modèle à 2 variables explicatives} 
On considère le modèle de régression $y_{i}=\beta_{1}+\beta_{2} x_{i, 2}+\beta_{3} x_{i, 3}+\varepsilon_{i}, \quad 1 \leq i \leq n$, que l'on écrit sous la forme $Y=X \beta+\varepsilon$. Les $x_{i, j}$ sont des variables exogènes du modèle, les $\varepsilon_{i}$ sont des variables aléatoires indépendantes, de loi normale centrée admettant la même variance $\sigma^{2}$. On a observé :

$$
X^{T} X=\left[\begin{array}{ccc}
30 & 20 & 0 \\
20 & 20 & 0 \\
0 & 0 & 10
\end{array}\right], \quad X^{T} Y=\left[\begin{array}{c}
15 \\
20 \\
10
\end{array}\right], \quad Y^{T} Y=59.5
$$

\begin{enumerate}
  \item Déterminer $n$, la moyenne des $x_{i, 3}$, le coefficient de corrélation des $x_{i, 2}$ et des $x_{i, 3}$.

  \item Estimer $\beta_{1}, \beta_{2}, \beta_{3}, \sigma^{2}$ par la méthode des moindres carrés ordinaires.

  \item Calculer pour $\beta_{2}$ un intervalle de confiance à $95 \%$ et tester l'hypothése $\beta_{3}=0.8$ au niveau $10 \%$.

%  \item Tester $\beta_{2}+\beta_{3}=3$ contre $\beta_{2}+\beta_{3} \neq 3$, au niveau $5 \%$.

  \item Calculer $\bar{y}$ et déduire le coefficient de détermination ajusté $R_{a}^{2}$.

  \item Construire un intervalle de prévision à $95 \%$ de $y_{n+1}$ si $x_{n+1,2}=3$ et $x_{n+1,3}=0.5$.

\end{enumerate}


\section{Un modèle à 3 variables explicatives}

On considère un modèle de régression de la forme :

$$
y_{i}=\beta_{1}+\beta_{2} x_{i, 2}+\beta_{3} x_{i, 3}+\beta_{4} x_{i, 4}+\varepsilon_{i}, \quad 1 \leq i \leq n .
$$

Les $x_{i, j}$ sont supposées non aléatoires. Les erreurs $\varepsilon_{i}$ du modèle sont supposées aléatoires indépendantes gaussiennes centrées de même variance $\sigma^{2}$. On pose comme d'habitude :

$$
X=\left[\begin{array}{cccc}
1 & x_{1,2} & x_{1,3} & x_{1,4} \\
\vdots & \vdots & \vdots & \\
1 & x_{n, 2} & x_{n, 3} & x_{n, 4}
\end{array}\right], \quad Y=\left[\begin{array}{c}
y_{1} \\
\vdots \\
y_{n}
\end{array}\right], \quad \beta=\left[\begin{array}{c}
\beta_{1} \\
\beta_{2} \\
\beta_{3} \\
\beta_{4}
\end{array}\right]
$$

Un calcul préliminaire a donné

$$
X^{\prime} X=\left[\begin{array}{cccc}
50 & 0 & 0 & 0 \\
0 & 20 & 15 & 4 \\
0 & 15 & 30 & 10 \\
0 & 4 & 10 & 40
\end{array}\right], \quad X^{\prime} Y=\left[\begin{array}{c}
100 \\
50 \\
40 \\
80
\end{array}\right], \quad Y^{\prime} Y=640
$$

On admettra que

$$
\left[\begin{array}{ccc}
20 & 15 & 4 \\
15 & 30 & 10 \\
4 & 10 & 40
\end{array}\right]^{-1}=\frac{1}{13720}\left[\begin{array}{ccc}
1100 & -560 & 30 \\
-560 & 784 & -140 \\
30 & -140 & 375
\end{array}\right]
$$

\begin{enumerate}
  \item Calculer $\hat{\beta}$, estimateur des moindres carrés de $\beta$, la somme des carrés des résidus $\sum_{i=1}^{50} \hat{\varepsilon}_{i}^{2}$, et donner l'estimateur de $\sigma^{2}$.

  \item Donner un intervalle de confiance pour $\beta_{2}$, au niveau $95 \%$. Faire de même pour $\sigma^{2}$ (on donne $c_{1}=29$ et $c_{2}=66$ pour les quantiles d'ordre $2,5 \%$ et $97,5 \%$ d'un chi-deux à $46 \mathrm{ddl}$ ).

  \item Tester la "validité globale" du modèle $\left(\beta_{2}=\beta_{3}=\beta_{4}=0\right.$ ) au niveau $5 \%$ (on donne $f_{46}^{3}(0,95)=$ 2.80 pour le quantile d'ordre $95 \%$ d'une Fisher à $(3,46) \mathrm{ddl}$ ).

  \item On suppose $x_{51,2}=1, x_{51,3}=-1$ et $x_{51,4}=0,5$. Donner un intervalle de prévision à $95 \%$ pour $y_{51}$.

\end{enumerate}
\section{Revenus européens}
Le tableau ci-dessous représente l'évolution du revenu disponible brut et de la consommation des ménages en euros pour un pays donné sur la période 1992-2001. [Pour les calculs, prendre 4 chiffres après la virgule].

\begin{center}
\begin{tabular}{|c|c|c|}
\hline
Année & Revenu & Consommation \\
\hline
1992 & 8000 & 7389.99 \\
\hline
1993 & 9000 & 8169.65 \\
\hline
1994 & 9500 & 8831.71 \\
\hline
1995 & 9500 & 8652.84 \\
\hline
1996 & 9800 & 8788.08 \\
\hline
1997 & 11000 & 9616.21 \\
\hline
1998 & 12000 & 10593.45 \\
\hline
1999 & 13000 & 11186.11 \\
\hline
2000 & 15000 & 12758.09 \\
\hline
2001 & 16000 & 13869.62 \\
\hline
\end{tabular}
\end{center}

On cherche à expliquer la consommation des ménages (C) par le revenu ( $\mathrm{R})$, soit :

$$
C_{t}=\alpha+\beta R_{t}+u_{t}
$$

\begin{enumerate}
\item Tracer le nuage de points et commenter.
%
\item Estimer la consommation autonome et la propension marginale à consommer $\hat{\alpha}$ et $\hat{\beta}$.
%
\item En déduire les valeurs estimées $\hat{C}_{t}$ de $C_{\mathrm{t}}$.

\item Calculer les résidus et vérifier la propriété selon laquelle la moyenne des résidus est nulle.

\item Calculer l'estimateur de la variance de l'erreur.

%\item Tester la significativité de la pente.

\item  Construire l'intervalle de confiance au niveau de confiance de $95 \%$ pour le paramètre $\beta$.

\item  Calculer le coefficient de détermination et effectuer le test de Fisher permettant de déterminer si la régression est significative dans son ensemble.

%\item Écrire et vérifier l'équation d'analyse de la variance. Interpréter.
%
%\item Après un travail minutieux, un étudiant de L3 trouve le coefficient de corrélation linéaire entre $C_{t}$ et $R_{t}$ suivant $r_{X Y}=0.99789619$. Sans le moindre calcul, tester la significativité de ce coefficient. Argumenter.

\item En 2002 et 2003, on prévoit respectivement 16800 et 17000 euros pour la valeur du revenu. Déterminer les valeurs prévues de la consommation pour ces deux années, ainsi que l'intervalle de prévision au niveau de confiance de $95 \%$.

\end{enumerate}
\section{De pères en fils}
Les données statistiques ci-dessous portent sur les poids respectifs des pères et de leur fils aîné.

\begin{center}
\begin{tabular}{|l|l|l|l|l|l|l|l|l|l|l|l|l|}
\hline
Père & 65 & 63 & 67 & 64 & 68 & 62 & 70 & 66 & 68 & 67 & 69 & 71 \\
\hline
Fils & 68 & 66 & 68 & 65 & 69 & 66 & 68 & 65 & 71 & 67 & 68 & 70 \\
\hline
\end{tabular}
\end{center}

\begin{enumerate}
\item Calculer la droite des moindres carrés du poids des fils en fonction du poids des pères.

\item Calculer la droite des moindres carrés du poids des pères en fonction du poids des fils.

\item Que vaut le produit des pentes des deux régressions?

\item Juger de la qualité des ajustements des questions précédentes.

\end{enumerate}
\section{Reconnaître un modèle de régression linéaire}
Les modèles suivants sont-ils des modèles de régression linéaire? Si non, peut-on appliquer une transformation pour s'y ramener? Pour chaque modèle de régression linéaire du type $Y=X \beta+\epsilon$, on précisera ce que valent $Y, X, \beta$ et $\epsilon$.

\begin{enumerate}
  \item On observe $\left(x_{i}, y_{i}\right), i=1, \ldots, n$ liés théoriquement par la relation $y_{i}=a_{0}+a_{1} x_{i}+$ $\epsilon_{i}, i=1, \ldots, n$. où les variables $\epsilon_{i}$ sont centrées, de variance $\sigma^{2}$ et non-corrélées. On désire estimer $a_{0}$ et $a_{1}$.

  \item On observe $\left(x_{i}, y_{i}\right), i=1, \ldots, n$ liés théoriquement par la relation $y_{i}=a_{1} x_{i}+a_{2} x_{i}^{2}+$ $\epsilon_{i}, i=1, \ldots, n$. où les variables $\epsilon_{i}$ sont centrées, de variance $\sigma^{2}$ et non-corrélées. On désire estimer $a_{1}$ et $a_{2}$.

  \item On relève pour différents pays $(i=1, \ldots, n)$ leur production $P_{i}$, leur capital $K_{i}$, leur facteur travail $T_{i}$ qui sont théoriquement liées par la relation de Cobb-Douglas $P=$ $\alpha_{1} K^{\alpha_{2}} T^{\alpha_{3}}$. On désire vérifier cette relation et estimer $\alpha_{1}, \alpha_{2}$ et $\alpha_{3}$.

  \item Le taux de produit actif $y$ dans un médicament est supposé évoluer au cours du temps $t$ selon la relation $y=\beta_{1} e^{-\beta_{2} t}$. On dispose des mesures de $n$ taux $y_{i}$ effectués à $n$ instants $t_{i}$. On désire vérifier cette relation et estimer $\beta_{1}$ et $\beta_{2}$.

  \item Même problème que précédemment mais le modèle théorique entre les observations s'écrit $y_{i}=\beta_{1} e^{-\beta_{2} t_{i}}+u_{i}, i=1, \ldots, n$, où les variables $u_{i}$ sont centrées, de variance $\sigma^{2}$ et non-corrélées.

\end{enumerate}

%
\end{document}
